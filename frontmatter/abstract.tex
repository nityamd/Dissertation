% the abstract

%!TEX root = ../dissertation.tex
% the abstract
\doublespacing
One of the crucial unsolved puzzles in astronomy has been the question of how galaxies end the process of star formation. A key piece in this puzzle is figuring out how to robustly infer star formation rates and the cosmic star formation histories of galaxies from observed photometric and spectroscopic data. The star formation history of a galaxy is encoded in its spectrum, which can be thought of as a fingerprint of a galaxy. However, spectroscopic information is often unavailable as it is more expensive to obtain. And where available, it can be limited in terms of interpretability by the dependence on the fiber diameter used and the physical properties of the galaxy in question.\\

The class of techniques known as SED (Spectral Energy Distribution) fitting tackles this problem by inferring the spectrum of a galaxy from photometric and spectroscopic data. Over the last decade, SED-fitting methods have become increasingly more sophisticated both in terms of model predictions as well as accounting for effects of more complex physical mechanisms such as dust absorption/emission. Another significant development, has been the rapidly growing field of IFU (Integral Field Unit) spectroscopy with a view to obtaining spatially resolved spectra with high S/N ratio of galaxies. With the advent of ongoing surveys such as MaNGA (Mapping Nearby Galaxies at Apache Point Observatory), the largest IFU-based survey thus far, we are better poised to answer this question than ever before.\\

In this dissertation, I examine these two techniques for estimating star formation histories. First, using data from NASA Sloan Atlas (NSA) catalog along with the Wide-field Infrared Survey (WISE), I compare star formation rates obtained from two different methods: one, a UV-to-IR SED fitting method that accounts for dust and the other, a  purely UV photometry-based approach. Using galaxy environments as a third independent parameter, I find a population of dust obscured star formers that masquerade as much lower star formation galaxies when only UV-optical information is available, but which live in the same large scale environments as other star forming galaxies.\\

In the second part, I examine the robustness of star formation history measurements from one of the largest available and most influential catalogs available, the Sloan Digital Sky Survey (SDSS) Legacy Survey. This catalog is limited by the fact that it uses small aperture fibers (3 arcsecond diameter) to measure galaxies that can sometimes have much larger angular extent. Spatially resolved spectroscopy from MaNGA measures galaxies more completely, but for a much smaller sample. I use aperture measurements of key spectral indicators of age, stellar mass and star formation history such as the $H \delta_{\rm A}$ absorption line index and the $D_{\rm n}4000$ break to quantify the effects of the small fiber aperture on the Legacy Survey sample results. From these results I show that biases exist in the measurements of stellar mass from the SDSS Legacy Survey, but that for most galaxies these biases are small.\\