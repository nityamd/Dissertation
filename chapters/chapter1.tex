\chapter{UV to IR Star Formation Indicators and Environments}
\label{ch:sfrk}

\newpage
\section{Introduction}
Determining the formation history of galaxies is one of the most important tasks in astronomy. The average star formation rates of galaxies have reduced with cosmic time (as reviewed by \citealt{madau_cosmic_2014}) and star-forming galaxies ``transition" to red and quiescent ones over that time \citep[\emph{e.g.}][among others]{peng_mass_2010, ilbert_mass_2013, muzzin_evolution_2013, moustakas_primus_2013, tomczak_galaxy_2014}. Many mechanisms have been proposed to explain the observed patterns of galaxy evolution but there are many unresolved questions regarding them. These mechanisms are known to have a some correlation with the environment, due to the fact that galaxy type depends on environment in the present day \citep{dressler_galaxy_1980, blanton_physical_2009-1}.\\

Star formation rate estimates are an important tool in the observational study of this problem \citep{kennicutt_star_2012}. The light emitted from galaxies can reveal recent star formation through the ultraviolet light of massive stars, recombination line emission in the ionized gas around those stars, in infrared light from dust in the galaxies absorbing the ultraviolet starlight, and from radio synchrotron emission from the ionized gas. Ultraviolet, optical, and infrared images and spectra are available for numerous low redshift galaxies, from the Sloan Digital Sky Survey (SDSS; \citealt{york00a}), the Wide-field Infrared Survey Explorer (WISE; \citealt{wright_wide-field_2010}), and the Galaxy Evolution Explorer (GALEX; \citealt{martin05a}).

In this chapter we compare two specific ways to estimate star formation rates from imaging photometry. We concentrate on imaging because it usually provides the most complete census of the galaxy; spectroscopy is often of only part of the galaxy, usually the center. We are interested specifically in comparing ultraviolet-based star formation indicators with indicators that account for the infrared light. The main difference between these types of indicators is how they account for dust absorption. The former ostensibly are corrected for dust absorption, but we generally expect that the latter should be more reliable and complete. However, detectable infrared light is not always available.

The presence of dust is an important complication in any analysis of galaxy light. Dust absorbs preferentially in the UV/optical region of the galaxy spectrum and both extincts and reddens the photometric data. The extinction in the visual bandpasses in star forming galaxies is highly variable but is commonly of order 50\%. This energy is re-emitted in the mid- to far-IR by poly-cyclic aromatic hydrocarbon molecules as well as warm and cold dust grains. Recent studies (\citealt{burgarella_herschel_2013}) find that in the nearby universe, almost $70\%$ of the FUV luminosity is obscured by dust on an average. Although UV estimates of star formation do attempt to account for dust attenuation (\citet{salim_uv_2007-1} ), the extensive reliance on UV SFR in the literature \citep[\emph{e.g.}][among others]{peng_mass_2010, moustakas_primus_2013, karim_star_2011, lee_comparison_2009, wyder_uvoptical_2007} makes it important to understand how well the dust corrections for UV SFR estimates work relative to other methods of estimating SFR.\\

Here, we examine a sample of galaxies whose UV-IR photometry is available and estimate the star formation rate in two independent ways. First we exploit the fact that we have UV to IR photometry and perform SED fitting using MAGPHYS \citep{da_cunha_simple_2008}, which accounts for dust by using a simple method of energy balance to obtain the specific star formation rates (SSFRs). The other method involves using purely UV photometry to estimate both star formation rate and dust attenuation using the prescription given by Salim et al \citep{salim_uv_2007-1}. We also estimate the environments of our population. We ask the following questions of this sample. How do these different star formation estimators disagree with each other? How does each estimate correlate with environment? Does one trace environment more closely? Under the assumption that the environment primarily correlates with the SSFR rather than dust obscuration's effect on the observables, studying this correlation will yield insights into the relative accuracy of the methods.\\

\section{Constructing a local sample spanning Ultraviolet to Infrared Imaging}
The sample on which we perform our measurements is based on the NASA-Sloan Atlas, a nearby galaxy sample which includes optical and ultraviolet imaging from SDSS and GALEX. We use NSA version {\tt v0\_1\_2}, whose redshift range extends up to $z = 0.055$ and includes SDSS (Data Release $9$) and GALEX imaging for galaxies. The total number of objects in this NSA catalog is ${\sim}150,000$. We define two samples from the NSA, one that we use to define environment (Environment Determining Population or EDP, hereafter) and one for which we determine star formation rates (Star Formation Rate Population or SFRP hereafter). We note that the SFRP is a subsample of the EDP. To get our EDP, we impose a volume-limited cut on this sample by retaining only galaxies with $-24.5 <M_{r}< -18.5$, which leaves us with $95,638$ galaxies.\\

\begin{figure}
\includegraphics[width=\textwidth]{figures/sample}
\caption[Short figure name.]{The Local sample distribution across optical and IR colors
\label{fig:myInlineFigure}}
\end{figure}

The star formation rate determinations we use here require both UV data and IR data. About 80\% of our galaxies have UV data from GALEX analyzed in the NSA. To obtain infrared imaging for the data we use the Wide-Field Infrared Survey Explorer (WISE) data and match the objects from the NSA with the ALLWISE Source Catalog to get the four-band infrared fluxes for each galaxy. About 99\% of galaxies in this sample have WISE data. 
We intend to study the evolution of these galaxies through the 
optical-IR color space, and therefore remove all galaxies with 
missing/faulty photometry in GALEX, SDSS, or WISE. This process 
includes limiting the color range of objects in the optical 
($7.5> N-r >=0$) and infrared ($3.0> [W1]- [W3] >= -3.0$) 
colors (top left corner of Fig. $1$). 
In addition, we remove some objects near the edges of the sample,
which do not have reliable environment determinations (see Section 
\ref{edge}). Our SFRP ultimately consists of $61,046$ objects with 
absolute magnitudes in the UV ($F$ and $N$ bands), optical 
($u$, $g$, $r$, $i$ and $z$ bands) and infrared ($W1$, $W2$, $W3$ 
and $W4$ bands).\\

\section{Estimating the Specific Star Formation Rates}

We compare two approaches to calculating specific star 
formation rates (SSFRs). The first uses UV, optical, and 
IR data. The second uses only the UV data.

\subsection{SED Fitting - MAGPHYS}

We develop here a method to quickly estimate SSFRs based 
on UV-optical and infrared colors. We begin by sorting galaxies 
into bins of the ($N-r$) - ($[W1]-[W3]$) color space 
($25\times25$ bins, where the optical and IR bin sizes are 
0.29 and 0.24 in magnitudes respectively). Within each bin, 
we normalize the fluxes of each galaxy relative to a constant 
flux in the $r$-band, and then take the mean normalized flux 
in each band over all galaxies in the bin. This procedure 
yields a ``template" SED for each bin in the color-color space.\\

We then use Multi-wavelength Analysis of Galaxy Physical Properties 
(\citet{da_cunha_simple_2008}) software to infer the SSFR for 
each template SED. MAGPHYS is a physical model-based method
to interpret the mid- and far-infrared spectral energy distributions 
of galaxies, self-consistently modeling the emission at ultraviolet, 
optical, and near-infrared wavelengths. For every input galaxy 
with a set of observed fluxes in different bands, MAGPHYS generates 
an optical and infrared library at that redshift and then 
samples all template spectra whose fluxes obey a simple principle 
of energy balance: that the amount of energy absorbed by dust 
in the UV/Optical matches the amount of infrared emission that 
is accounted for purely by dust. Once the templates have been 
sub-sampled thus, MAGPHYS uses chi-squared fitting to see which 
combination best reproduces the observed fluxes along with the 
likelihood for the distributions. The results for the SSFRs 
thus obtained for each template are shown in the lower left 
panel of Fig $1.2$ along with the number distribution of 
the galaxies across the chosen bins. Note that bins with 
$< 5$ galaxies were omitted as those regions of the 
color-color space are obviously under-sampled.\\ 

The resulting distribution of the MAGPHYS-based SSFRs 
across the color-color space looks as we would expect it 
to for the most part. In UV-optical colors, blue galaxies 
have high SSFRs, and red galaxies have low SSFRs. However, 
there is also a dependence of SSFR on IR color; most notably, 
in the UV-optical green valley, the redder galaxies in the IR 
have higher SSFRs. These galaxies are likely to be 
dust-extincted in the UV, with the reemission by dust 
reddening the IR colors. In what follows, we will use the 
UV-optical and IR colors of individual galaxies, and the 
dependence of the SSFR on these colors shown in the lower 
left panel of Fig $1.2$, to assign an SSFR to each 
individual galaxy.\\

\subsection{UV Star Formation Rates}

We also explore a simple method of determining star formation 
rates developed by \citet{salim_uv_2007-1} that depends only on 
the UV fluxes of each galaxy. It assigns a star formation rate that 
is proportional to the UV luminosity (specifically the FUV band 
if we are looking at GALEX). Dust attenuation is also accounted 
for in this method by looking at the ratio of luminosities in 
the FUV and NUV bands.\\

According to this prescription \citep{salim_uv_2007-1}, 
the star formation rate is given by:
$$ SFR = 1.08 \times 10^{-28}{L^{0}}_{\rm FUV}, $$
where ${L^{0}}_{FUV}$ is the rest-frame FUV luminosity. This method accounts for dust attenuation of the FUV light as well by estimating an attenuation factor $A\nu$ as follows.\\

If $N-r \geq 4.0$, i.e. for the red sequence galaxies,\\

$$ A\nu = \begin{cases} 3.32 (F-N) + 0.22, & \text{if} (F-N) < 0.95\\3.37, & \text{if} (F-N) \geq 0.95. \end{cases}$$

If $N-r < 4.0$ , i.e. for the blue sequence galaxies,\\

$$A\nu = \begin{cases} 2.99(F-N) + 0.27, & \text{if}(F-N) < 0.90\\2.96, & \text{if} (F-N) \geq 0.90. \end{cases}$$\\

This method uses the UV slope to estimate the dust attenuation. The 
physical idea is that when the UV light is dominated by massive young
stars, these are hot enough that we are observing their Rayleigh-Jeans
tail spectra, and this slope is basically constant with age of the
population. Thus the observed UV color reveals the dust reddening.

The lower right panel of Fig. $1.2$ shows the mean 
SSFR\textsubscript{UV} as a function of color. Compared to the 
lower left panel, the SSFR\textsubscript{UV}'s have a nearly 
monotonic relationship with UV-optical color whereas, as discussed 
above, the SSFR\textsubscript{MAGPHYS}'s do not. Assuming 
the SSFR\textsubscript{MAGPHYS}'s are closer to correct, the 
UV-optical green valley contains a population of galaxies that 
are not truly transitioning but instead are reddened by the 
presence of dust. Furthermore, the nominally dust-corrected 
UV star formation rates do not successfully identify these 
galaxies. 

\textbf{MRB: Quantify the fraction of galaxies like this; say 
within $2.5 < N-r < 4.5$, what fraction have $[W1]-[W3] > 1.5$, 
answering same question in 3 bins of absolute magnitude too.}
\\

\section{Environments}

We identified above a population of galaxies isolated in 
color space, which appeared to have dust-extincted star formation, 
such that the SSFR\textsubscript{UV} estimate was much less 
than the SSFR\textsubscript{MAGPHYS}. In order to confirm whether 
MAGPHYS is capturing an inherent property in this population of 
galaxies, we examine an independent physical property, namely 
the environments of our sample.\\

\begin{figure}
\includegraphics[width=\textwidth]{figures/1_panel_plot.pdf}
\caption[Short figure name.]{The SSFR measurements and environment 
    shown the color-color 
    space. Each point in the plot is shown at the mean colors in each of 
    the bins we use. The grey value or color of the points in each
    panel show the mean value in each bin for the quantity described 
    by the corresponding  color bar.
    \emph{Top left:} Logarithmic number density in each of the 
    bins; all bins with less than $5$ galaxies were discarded in this 
    and the other panels. \emph{Top right:} Environment; in each bin, 
    the average number of nearest neighbors is calculated in a 
    projected cylinder ($r_{t} = 0.5 Mpc$ and $v_{los} = \pm 1000 km/s$). 
    \emph{Bottom left:} the Specific Star Formation Rates obtained from 
    MAGPHYS. \emph{Bottom right:} UV Specific Star Formation Rates 
    estimated by using the method described in \citet{salim_uv_2007}.
\label{fig:myInlineFigure}}
\end{figure}

\subsection{Measures of Environment}

The environment of a galaxy can be defined in many ways, such as 
fixed aperture counts, distance to the $n^{th}$ nearest neighbor, 
Voronoi volumes, etc \citep{cooper_measuring_2005}. Here we use counts 
in a projected fixed aperture of radius $0.5$ Mpc as our environment 
measure. Around every galaxy we construct a projected cylinder with a 
radius (in the transverse direction) of $r_{t} = 0.5$ Mpc and a line 
of sight velocity window of $v_{los} = \pm 1000 km/s$. For each
galaxy for which we have star formation measurements, we count its 
number of neighbors ($n_{\rm env}$) in this cylinder(Fig. $1.2$) in
the Environment Defining Population.\\

\subsection{Edge Effects}\label{edge}

We must also account for the survey edges. For galaxies at the edge, 
part of the fixed aperture used to estimate the environments might 
lie outside the survey coverage. It is important to identify these 
galaxies and either discard them or assign an appropriate weight to 
$n_{\rm env}$ in order to account for the missing area. 

To identify the edges, we use (Swanson et. al.'s) \texttt{Mangle}, 
a suite of free open-source software designed to deal with complex 
angular masks in an efficient and accurate manner. First, the 
NYU-VAGC \texttt{Mangle}-format mask was used to obtain the angular 
mask for the NASA  Sloan Atlas by using the \emph{polyid} routine 
from \texttt{Mangle}. Then, the \emph{ransack} routine was used to 
populate the mask with a random sample of $N = 10,000,000$ galaxies. 
For each galaxy, we compute the angular separation $\theta_{i}$ 
that corresponds to our $0.5 Mpc$ aperture at the redshift 
of that galaxy. We then count the number of galaxies $n_{i}$ that 
lie within this angular separation and compare the value obtained 
to the expected value: 
$$\big \langle n \big \rangle _{i} = \frac{N}{A_{\mathrm{EDP}}} \times \pi \theta_{i}^{2} \times f_{\mathrm{thresh}} $$
$A_{\mathrm{EDP}}$ is the total area of the mask and $f_{\mathrm{thresh}}$ is the fractional threshold for the edge effect cut-off. In our case, we chose $f_{\mathrm{thresh}} = 0.8$. Wherever $n_{i} < \big \langle n \big \rangle _{i} $, we consider the galaxy to be near an edge and discard it from our sample. \\

\subsection{Environments in Color-Color Space}

The mean environment, quantified by $(\langle n_{\rm env}\rangle + 1)$
in each bin are shown in the upper right panel of Fig. $1.2$ 
as a function of optical and infrared colors. As we expect from 
many previous studies (\citealt{blanton_physical_2009}) 
the star-forming bluer galaxies tend  to exist in less dense 
environments on average and the  red-and-dead population tends to 
exist in more dense environments.  In the UV-optical green valley 
region ($3 < N-r < 5$) there is some indication that the 
environment declines with $[W1]-[W3]$ color at fixed UV-optical 
color. However, the mean environments in this plot cannot easily 
be interpreted, because the mean stellar mass in each bin is 
different, so some of the variation is driven by the dependence 
of environment on stellar mass.

\begin{figure}
\includegraphics[width=\textwidth]{figures/2_env_plot.pdf}
\caption[Short figure name.]{The two star formation rate estimates (from Fig.$1.2$) plotted against each other as a function of the environment; We notice two distinct set of outliers that seem to have lower UV SSFR's but similar environments to the galaxy bins with the same MAGPHYS SSFR's.
\label{fig:myInlineFigure}}
\end{figure}

\section{The Environments of the Outliers}

Fig. $1.3$ shows the relationship between the two SSFR measurements,
with points colored according to the mean environment, for each bin 
in color-color space from Fig. $1.2$. There is a set of bins in 
the range $-10.5 < \mathrm{SSFR\textsubscript{MAGPHYS}}< -9.5$ that 
have lower SSFR\textsubscript{UV} values than the general trend, 
and have environments similar to the other galaxy bins in the same 
MAGPHYS range. Hereafter we shall refer to this population as the 
``Outliers," to distinguish them from the general trend line in 
Fig. $1.3$. 

We identify these outliers more specifically in Fig. $1.4(a)$, as 
being in the lower right square. The galaxy bins with the same 
SSFR\textsubscript{MAGPHYS} values (above the square) are identified 
as the ``MAGPHYS Box" and the bins with the same SSFR\textsubscript{UV} 
values (left of the square) are identified as ``UV BOX" in Fig. 
$1.4(a)$. \\

.. change line styles of different boxes in Fig.$1.5$ so that the result can be inferred on b-and-w print...
  
\subsection{Green-valley interlopers}

When we return to the color-color space (Fig. $1.4(b)$) and 
show where the bins in each box lie, we see the trends we would expect. 
The Outliers lie in a similar UV-optical color range as the UV box but 
have higher IR color. The MAGPHYS Box occupies the bluer side 
in the optical color range while spanning almost the entire IR 
color range. The UV Box occupies the redder side in the optical 
color range while having lower IR color values. The Outlier bins 
are the same bins we previously identified as the galaxies in the 
UV-optical ``green valley" that are there due to dust reddening. 

To verify this, we unwrap the bins and look at the Probability 
Density Function of the Environments of the galaxies in these three regions in the mass range  $ 9.5 < M_{*} < 10.7$ and examine the distribution of environments in these three regions, the result of which is plotted in Fig. $1.5$.

\begin{figure}
\includegraphics[width=\textwidth]{figures/3_outliers.pdf}
\caption[Short figure name.]{The outliers shown as a function of the Star Formation Rates as well as Optical and IR colors
\label{fig:myInlineFigure}}
\end{figure}
 
 \subsection{Jackknife Errors}
 To calculate uncertainty in the estimated probability density functions of the environments, $P_{i,\mathrm{bin}}$'s : $i = 1,2,3$ for each of the populations, we use the standard jackknife technique. Jackknife re-sampling gives us an internal error estimate that tests how representative a measurement/trend is of the data it is estimated with. We divide our entire sample into $20$ subsamples with nearly equal co-moving volumes and estimate the same probability density functions($P^{j}_{i}$'s) for the whole sample while leaving out one subsample each time. We can then estimate our uncertainty for each bin in the PDF's thus:\\
 $$ \sigma_{i, \mathrm{bin}} = \sqrt{\frac{N}{N-1} \sum_{j = 1}^{j = N} (P^{j}_{i} - P_{i,bin})^{2}} $$
 The errors estimated in this manner account for Poisson shot noise and also the sample variance, the extra error associated with the fact that the density field varies across the survey. Although precision studies of large scale structure have found that latter effect is not perfectly accounted for with standard jackknife techniques, they are precise enough for our purposes here. The jackknife errors are shown in Fig. $4$ and confirm our hypothesis: that the Outliers are a little different than the MAGPHYS Box but still, very different from the UV Box.\\

\begin{figure}
\includegraphics[width=\textwidth]{figures/4_jk_plot.pdf}
\caption[Short figure name.]{Probability density functions of the Environments of the three populations described in Fig. $2$
\label{fig:myInlineFigure}}
\end{figure}

\section{Summary and Conclusion}
 \begin{itemize}
 \item{From Fig $1$, we see that SSFR\textsubscript{MAGPHYS} identifies a region in the color-color space as dust-obscured star forming galaxies and correlates better with the environments of the galaxies.}
 \item{At the higher star formation end, we find that the dust-obscured star-formers as identified by MAGPHYS have environments comparable to the blue star-forming galaxies, confirming that this is indeed a physical effect we're seeing.}
 \item{Comparing the environment distribution of the Outliers relative to the galaxies with (a) the same SSFR\textsubscript{MAGPHYS}'s as the Outliers and (b) the same SSFR\textsubscript{UV}'s as the Outliers (Fig. $1.5$), we find that the Outliers indeed have a similar environment distribution to the galaxies that have the same SSFR\textsubscript{MAGPHYS}'s, i.e., they seem to favor lower environment densities mimicking the behavior of star-formers.}
 \end{itemize}
 
 