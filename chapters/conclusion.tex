\chapter{Conclusion}
\label{conclusion}

In this thesis I have explored two systematic issues affecting the study of star formation histories of galaxies. In Chapter \ref{ch:sfrk}, I examined a commonly used method of star formation estimation that is purely dependent on UV emission and assessed how well it traces star formation in dusty galaxies. In Chapter \ref{ch:acm}, I evaluated the effects of aperture bias on the stellar masses estimated in the largest star formation and stellar mass catalog of galaxies, the MPA-JHU catalog from SDSS-I and -II. The need for calibrating observational indicators of star formation histories and investigating the systematic errors they might have has been discussed in Chapter \ref{introduction}. In this section I will summarize and discuss the significance of both these results, where they would be applicable and the scope for further exploration in either area.\\

The significant results in Chapter \ref{ch:sfrk} are that the UV based star formation rates are not comprehensive tracers of star formation, especially in the case of dusty star formers that reside in the optical ``green valley." We need estimates that incorporate IR emission to help us identify these dusty star formers much better. These conclusions were arrived at by looking at a purely UV-based star formation rate and a UV-to-IR based star formation rate for a sample of nearby galaxies. The latter was used to identify a distinct population of dust-obscured star formers in color space. To confirm this identification, a third independent property, namely the environment of the galaxies was estimated and it was found that the environments of this population behaved more in alignment with star forming galaxies than passive, red and dead galaxies.\\

It has been understood for decades that UV-only star formation
indicators are incomplete. Many UV-based estimators, including
the one tested here, attempt to correct for this effect. 
The most natural interpretation of a mismatch between
the UV indicators and IR-based indicators is that these 
corrections are not sufficient for all galaxies. Our work
here provides a new, model-independent piece of evidence 
that this interpretation is correct.

UV star formation indicators are particularly important
at high redshift, because the rest frame IR light becomes
harder and harder to access (see \citealt{madau_cosmic_2014} and 
references therein). The corrections for dust extinction
at high redshift form the basis of many estimates of the
star formation at those epochs (e.g. \citealt{overzier11a}). 
Our results here highlight the limitations of this approach
and the importance of new windows on the faint IR universe,
such as the upcoming James Webb Space Telescope 
(\citealt{gardner06a}).\\

In Chapter \ref{ch:acm}, I investigated the effect of 
aperture bias due to the 3$''$ SDSS fiber in the stellar 
mass estimation method for the MPA-JHU catalog. To do so,
I used spatially resolved spectral data from the MaNGA 
IFU-based survey. Using this data, I reproduced the 
\citet{kauffmann_stellar_2003} method, which relies on two 
key spectral indicators. I compared the measurement
of these indicators for the full galaxy aperture against 
the measurement for an equivalent 3$''$ aperture, to 
test how biased mass-to-light ratios would be if
estimated from the latter.

Although the mean offset in mass-to-light ratio
is fairly small, the typical dispersion of the offset
is fairly large (0.2--0.3 dex). This means that the 
stellar masses measured in MPA-JHU have a stellar mass
error due to this effect; because it is related to 
the spectral indicators, it implies that this error 
correlates with the galaxy structure. The level of 
error is large relative to estimates of the intrinsic
scatter in stellar mass at fixed halo mass (e.g., 
\citealt{tinker17a}) and thus is not ignorable. It is 
also large enough to affect estimates of the stellar
mass function at the massive end, where the slope with
mass is steep.

The results here potentially have implications for 
a large number of works based on the MPA-JHU catalog,
which has been used in hundreds of investigations.
Although our aperture effects may affect some investigations,
for example the above-mentioned measurements of the 
stellar mass function at high mass,
they may not be important to all of these investigations. 
For example, many investigations only use mass-to-light
ratios to bin galaxies coarsely into mass bins
(e.g. 0.5--1 dex in width), rendering
any aperture effect issues relatively small.
I provide, in Chapter \ref{ch:acm}, the basic numbers 
with which to evaluate the potential importance of 
aperture effects in considering results based on the 
MPA-JHU catalog.

Our method of estimating aperture corrections may be 
used for other quantities measured from SDSS-I and -II as
well, for example the star formation rates of \citet{brinchmann_physical_2004}, also found in the 
MPA-JHU catalog. This effect would also have
consequences for the interpretation of 
a large number of investigations.\\

Future surveys may also want to evaluate how sensitive 
they are to these effects as a function of redshift based
on MaNGA data. For example, the upcoming DESI Bright Galaxy
Survey \citep{desi16a} will survey millions of nearby
galaxies with many science goals in common with 
SDSS-I and -II, but for more galaxies at somewhat
higher redshifts. I have shown here that MaNGA can be used
to estimate what limitations might be imposed by aperture bias 
on that survey.\\

The topics in this thesis only touch on a few elements in 
the careful study of galaxy populations and the measurements
of their star formation history. 
Over decades, the community of astronomers has transformed
the study of galaxies into quantitative measurements
of their history accompanied by detailed comparisons to
theoretical models. Each of the results presented in this
thesis provides a new brick in the foundation of this 
community's understanding of galaxies in the universe.

