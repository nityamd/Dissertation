\chapter{Conclusion}
\label{conclusion}

In this thesis I have successfully explored two aspects of calibrating star formation indicators. In Chapter \ref{ch:sfrk}, I have examined one of the most popularly used methods of star formation estimation that is purely dependent on UV emission and assessed how well it traces star formation in dusty galaxies and in Chapter \ref{ch:acm}, I have evaluated the effects of aperture bias in the stellar masses estimated  at the largest star formation and stellar mass catalog of galaxies, the MPA-JHU catalog. The need for calibrating observational indicators of star formation indicators and investigating the systematic errors they might have has been discussed in Chapter \ref{introduction}. In this section I will summarize and discuss the significance of both these results, where they would be applicable and the scope for further exploration in either area.\\

The significant results in Chapter \ref{ch:sfrk} are that the UV based star formation rates are not efficient tracers of star formation, especially in the case of dusty star formers that reside in the optical ``green valley" and that we need IR based calibrations to help us identify these dusty star formers much better. These conclusions were arrived at by looking at a purely UV-based star formation rate and a UV-to-IR based star formation rate for a sample of nearby galaxies and the latter was used to identify a distinct population of dust-obscured star formers in a color-color phase space. To confirm this identification, a third independent property, namely the environment of the galaxies was estimated and it was found that the environments of this population behaved more in alignment with star forming galaxies than passive, red and dead galaxies.\\

Since the advent of the space-based GALEX telescope, most high redshift surveys have relied heavily on UV-based star formation rates. These UV-based measurements have played an essential sole in mapping the cosmic star formation history of the Universe. My result however implies that UV based star formation rates fail in a non-trivial way in capturing dusty star formers and tend to categorize them as less star forming than what they really are. There is further scope for exploration of this and the corrections this would entail in the near future with the launch of the James Webb Space Telescope (JWST) which aims at getting long wavelength visible to mid-IR information for high redshift objects.\\

In Chapter \ref{ch:acm}, I have investigated the effect of aperture bias due to the 3'' SDSS fiber in the stellar mass estimation method for the MPA-JHU catalog by making use of spatially resolved spectral data from the MaNGA IFU-based survey and reproducing the \citet{kauffmann_stellar_2003} method which relies to two key spectral indicators. And while the mean tendency in the offsets from full aperture measurements is small, the dispersion in the offsets across the phase space of the spectral indicators was found to be high, implying a correlation between the systematic error in stellar mass and a galaxy's morphology. In particular, it implies that the MPA-JHU method over-estimates the stellar mass of galaxies with strong bulges. Further, the implication of this for the halo masses inferred from stellar masses via abundance matching implies is a cause for concern while using the MPA-JHU masses.The method I used to investigate the aperture bias in the MPA-JHU masses can be similarly extended to the MPA-JHU catalog star formation rates as well. This form of aperture correction would be crucial for any fiber-based survey and in particular, for upcoming surveys like the DESI Bright Galaxy Survey, an exploration of the systematic effects of aperture bias on the estimation of galaxy properties in the same vein as mine can be immensely useful.\\

